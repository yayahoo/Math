\documentclass{exam}
\usepackage{mycommands,amsmath,amsfonts,tikz,xeCJK}
\setCJKmainfont{STSong}
\begin{document}
\begin{questions}
    \question
    \begin{parts}
        \part
        设$\gamma$是$\MC$中的可求长曲线,$g$是$\gamma$上的连续函数,定义
  \[
    G(z) = \frac1{2\pi i}\int\limits_\gamma\frac{g(\zeta)}{\zeta-z}\diff \zeta,
  \]
 在$\MC\backslash\gamma$上有任意阶导数,而且
  \begin{equation}\label{eq3.4.4}
    G^{(n)}(z) = \frac{n!}{2\pi i}\int\limits_\gamma \frac{g(\zeta)}{(\zeta-z)^{n+1}}\diff \zeta,n=1,2,\cdots.
  \end{equation}
  \part
  设$\gamma_0,\gamma_1,\cdots,\gamma_k$是$k+1$条可求长简单闭曲线,$\gamma_1,\cdots,\gamma_k$都在$\gamma_0$的内部,$\gamma_1,\cdots,\gamma_k$中的每一条都在其他$k-1$条的外部,$D$是由这$k+1$条曲线围成的域,$D$的边界$\gamma$由$\gamma_0,\gamma_1,\cdots,\gamma_k$所组成.如果$f\in H(D)\cap C(\bar D)$,则对任意$z\in D$,有
  \[f(z)=\frac1{2\pi\ii}\int\limits_{\gamma}\frac{f(\zeta)}{\zeta-z}\diff \zeta.\]
  $f$在$D$内有任意阶导数,且
  \[
    f^{(n)}(z) = \frac{n!}{2\pi\ii}\int\limits_\gamma\frac{f(\zeta)}{(\zeta-z)^{n+1}}\diff \zeta,n=1,2,\cdots.
  \]
    \end{parts}
    \newpage
    \question
    \begin{parts}
        \part
        级数收敛及其cauchy准则,一致收敛及其cauchy准则
        \part
        级数$\sum_{n=1}^\infty f_n(z)$在$E$上一致收敛的充要条件是对任意$\varepsilon>0$,存在正整数$N$,当$n>N$时,不等式
  \begin{equation}\label{eq4.1.3}
    |f_{n+1}(z) + \cdots + f_{n+p}(z)| < \varepsilon
  \end{equation}
  对所有$z\in E$及任意自然数$p$成立.
  \part
  幂级数,是指形如
\begin{equation}\label{eq4.2.1}
  \sum_{n=0}^\infty a_n(z-z_0)^n = a_0 + a_1(z-z_0) + \cdots + a_n(z-z_0)^n + \cdots
\end{equation}
的级数.
\part
  级数  \eqref{eq4.2.1} 存在收敛半径
  \[
    R = \frac1{\varlimsup\limits_{n\to\infty} \sqrt[\leftroot{-1}\uproot{2}n]{|a_n|}}.
  \]
    \end{parts}
\newpage
\question
若$f\in H\big(B(z_0,R)\big)$,则$f$可以在$B(z_0,R)$中展开成幂级数:
  \begin{equation}\label{eq4.3.1}
    f(z) = \sum_{n=0}^\infty \frac{f^{(n)}(z_0)}{n!} (z-z_0)^n, z\in B(z_0,R).
  \end{equation}
  右端的级数称为$f$的\textbf{Taylor级数}.
\end{questions}

\end{document}